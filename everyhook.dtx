% \iffalse The license starting three lines down applies to this file
%<*batchfile>
\begingroup\input docstrip \preamble
Copyright (C) 2010, 2011, 2014 by Stephen Checkoway

This file may be distributed and/or modified under the
conditions of the LaTeX Project Public License, either
version 1.3c of this license or (at your option) any later
version.  The latest version of this license is in:

    http://www.latex-project.org/lppl.txt

and version 1.3c or later is part of all distributions of
LaTeX version 2005/12/01 or later.

This work has the LPPL maintenance status 'maintained'.

The Current Maintainer of this work is Stephen Checkoway.

This work consists of everyhook.dtx and the derived file
everyhook.sty.
\endpreamble
\keepsilent
\usedir{tex/latex/everyhook}
\askforoverwritefalse
\generate{\file{\jobname.sty}{\from{\jobname.dtx}{}}}
\endgroup

\documentclass{ltxdoc}
\usepackage{fixltx2e}
\usepackage{etoolbox}
\usepackage[osf]{mathpazo}
\usepackage[scaled=0.86]{berasans}
\usepackage[scaled=1.03]{inconsolata}
\usepackage[T1]{fontenc}
\usepackage{microtype}
\usepackage[margin=1.5in]{geometry}
\usepackage[excludeor]{everyhook}
\usepackage{enumitem}
\usepackage[pdfborder={0 0 0}]{hyperref}
\usepackage{ifpdf}
\usepackage{listings}
\usepackage{lipsum}
\begingroup
\catcode`\^^09\active
\gdef\obeytabs{%
\catcode`\^^09\active
\def^^09{\ \ \ \ \ \ \ \ }%
}
\endgroup
\makeatletter
\patchcmd\macrocode{\@vobeyspaces}{\@vobeyspaces\obeytabs}{}{\undefined}
\makeatother

\CheckSum{190}
\CharacterTable
 {Upper-case    \A\B\C\D\E\F\G\H\I\J\K\L\M\N\O\P\Q\R\S\T\U\V\W\X\Y\Z
  Lower-case    \a\b\c\d\e\f\g\h\i\j\k\l\m\n\o\p\q\r\s\t\u\v\w\x\y\z
  Digits    \0\1\2\3\4\5\6\7\8\9
  Exclamation   \!     Double quote  \"     Hash (number) \#
  Dollar    \$     Percent       \%     Ampersand     \&
  Acute accent  \'     Left paren    \(     Right paren   \)
  Asterisk      \*     Plus      \+     Comma     \,
  Minus     \-     Point     \.     Solidus       \/
  Colon     \:     Semicolon     \;     Less than     \<
  Equals    \=     Greater than  \>     Question mark \?
  Commercial at \@     Left bracket  \[     Backslash     \\
  Right bracket \]     Circumflex    \^     Underscore    \_
  Grave accent  \`     Left brace    \{     Vertical bar  \|
  Right brace   \}     Tilde     \~}

\EnableCrossrefs
\CodelineIndex
\RecordChanges
\GetFileInfo{\jobname.sty}
\title{The \textsf{\jobname} package\thanks{This document
corresponds to \textsf{\jobname}~\fileversion, dated \filedate.}}
\author{Stephen Checkoway\\\texttt{s@cs.jhu.edu}}

\lstset{language=[LaTeX]TeX,basicstyle=\ttfamily,
texcsstyle=*\normalfont\bfseries,
moretexcs={paragraph,RequirePackage},xleftmargin=\parindent}

\renewcommand*\arg[1]{\textsf{\{#1\}}}
\makeatletter
\renewcommand*\meta@font@select{\normalfont\itshape}
% \DescribeOption taken from trivfloat.dtx
\newcommand*{\DescribeOption}{%
  \leavevmode
  \@bsphack
  \begingroup
    \MakePrivateLetters
    \Describe@Option
}
\newcommand*{\Describe@Option}[1]{%
    \endgroup
  \marginpar{%
    \raggedleft
    \PrintDescribeEnv{#1}%
  }%
  \@esphack
  \ignorespaces
}
\makeatother
\begin{document}
\maketitle

\begin{abstract}
The \textsf{\jobname} package takes control of the six \TeX\
token parameters \cs{everypar}, \cs{everymath}, \cs{everydisplay},
\cs{everyhbox}, \cs{everyvbox}, and \cs{everycr}. Real hooks for
each of these can be installed using a stack like interface. For
compatibility with \LaTeX\ standard classes and packages, each of
the \cs{everyX} token lists can be set without interfering with the
hooks.
\end{abstract}
\tableofcontents

\section{Introduction}
\TeX\ contains nine token parameters, seven of which are inserted
into the current list at various times. Quoting from \textit{The
\TeX book}, the seven token parameters of interest are\footnote{The remaining two
token parameters are \cs{output} and \cs{errhelp}.}
\begin{description}[labelindent=1em, noitemsep]
\item{\cs{everypar}\enskip} tokens to insert when a paragraph begins,
\item{\cs{everymath}\enskip} tokens to insert when math in text begins,
\item{\cs{everydisplay}\enskip} tokens to insert when display math begins,
\item{\cs{everyhbox}\enskip} tokens to insert when an hbox begins,
\item{\cs{everyvbox}\enskip} tokens to insert when a vbox begins,
\item{\cs{everyjob}\enskip} tokens to insert when the job begins, and
\item{\cs{everycr}\enskip} tokens to insert after every \cs{cr} or
nonredundant \cs{crcr}.
\end{description}
Of these, \cs{everyjob} is not very useful outside of INITEX and so
it won't be considered further.

The remaining six token parameters can be used to great effect. For
example, the \cs{everypar} is used in \cs{paragraph} to set the title
of the paragraph inline allowing constructions like\\
\indent\hbox to.45\textwidth{%
\begin{lstlisting}[boxpos=b]
\paragraph{Paragraph title.}

A blank line followed by
the rest of the paragraph.
\end{lstlisting}\hfil
}
\hfill
\parbox[b]{.45\textwidth}{
\paragraph{Paragraph title.}

\strut A blank line followed by the rest of the paragraph.
\\[\baselineskip]
}\\[\baselineskip]
which work properly rather than starting a new paragraph due to the
blank line.

Similarly, \cs{everymath} and \cs{everydisplay} are used by the
\LaTeX\ kernel to set up math fonts.

Using the \TeX\ primitives directly has the major downside that they
cannot be used by multiple packages at the same time. Setting
\cs{everypar} overwrites a prior usage. Even if one package is careful
and always uses
\begin{lstlisting}
\everypar=\expandafter{\the\everypar new tokens here}
\end{lstlisting}
so as not to stomp on another's usage, there's no guarantee that the
other package will not later set \lstinline!\everypar={}!.

To get around this, the \textsf{\jobname} package takes control of the
six \cs{everyX} primitives listed above and for each one provides a
stack like interface for two additional token lists, one to be
expanded before the \cs{everyX} and one to be expanded after. For
example,
\begin{lstlisting}
\PushPreHook{hbox}{1}
\PushPreHook{hbox}{2}
\everyhbox={3}
\PushPostHook{hbox}{4}
\PushPostHook{hbox}{5}
\end{lstlisting}
will cause the insertion of the tokens |21345| at the start of an
\cs{hbox}. Note that \cs{PushPreHook} adds tokens to the \emph{left}
of the list of tokens to appear before those in \cs{everyhbox} whereas
\cs{PushPostHook} adds tokens to the \emph{right} of the list of
tokens to appear after those in \cs{everyhbox}.

\section{Usage}
The \textsf{\jobname} package has one (rather experimental) option,
|excludeor| and is loaded using
\begin{lstlisting}
\usepackage[excludeor]{everyhook}
\end{lstlisting}
or
\begin{lstlisting}
\RequirePackage[excludeor]{everyhook}
\end{lstlisting}
as required where the option is, of course, optional.

\subsection{Options}
\DescribeOption{excludeor}
Some of the hooks described below can cause unwanted behavior when
active during the execution of \LaTeX's output routine. The
experimental |excludeor| option saves and clears the hooks at
the beginning of the output routine and restores them at the end.

\subsection{Manipulating hooks}
There are 12 hooks, a pre and post hook for each of the six token
parameters |par|, |math|, |display|, |hbox|, |vbox|, and |cr|. The
first argument to all of the macros described in this section must be
one of these six. \textbf{All hook manipulation is \emph{global}.}

\paragraph{Pre hooks.}
\DescribeMacro{\PushPreHook}
\DescribeMacro{\PopPreHook}
Additional tokens \meta{balanced text} are prepended to the pre hook
\meta{hook} using \cs{PushPreHook}\arg{\meta{hook}}\arg{\meta{balanced
text}}. The most recently pushed tokens can be popped off using
\cs{PopPreHook}\arg{\meta{hook}}.

\paragraph{Post hooks.}
\DescribeMacro{\PushPostHook}
\DescribeMacro{\PopPostHook}
Additional tokens \meta{balanced text} are appended to the post hook
\meta{hook} using
\cs{PushPostHook}\arg{\meta{hook}}\arg{\meta{balanced text}}. The most
recently pushed tokens can be popped off using
\cs{PopPostHook}\arg{\meta{hook}}.

\paragraph{Saving, restoring, and clearing hooks.}
\DescribeMacro{\SavePreHook}
\DescribeMacro{\SavePostHook}
\DescribeMacro{\RestorePreHook}
\DescribeMacro{\RestorePostHook}
\DescribeMacro{\ClearPreHook}
\DescribeMacro{\ClearPostHook}
Each of the 12 pre and post hooks can be saved to a macro, restored
from a macro, or cleared independently. To save the pre hook
\meta{hook} to the macro \cs{cs}, use
\cs{SavePreHook}\arg{\meta{hook}}\arg{\cs{cs}}. Restoring is
accomplished by \cs{RestorePreHook}\arg{\meta{hook}}\arg{\cs{cs}}. To
clear all of the tokens in a pre hook use
\cs{ClearPreHook}\arg{\meta{hook}}. The \cs{SavePostHook},
\cs{RestorePostHook}, and \cs{ClearPostHook} are analogous.

\section{Example}
As a nontrivial example of where this package can be used, consider
the following example.
\begin{lstlisting}
\documentclass{article}
\usepackage{everyhook}
\usepackage{lipsum}

\begin{document}
\setlength{\parindent}{0pt}
\PushPreHook{par}{\llap{\textbullet\enskip}\null}
\paragraph{Lorem ipsum.}
\lipsum[1-4]
\PopPreHook{par}
\end{document}
\end{lstlisting}
This code will cause each paragraph of the \emph{lorem ipsum} text to
have no indentation and instead to place a bullet in the margin. See
Figure~\ref{fig:example}. If \cs{everypar} were used instead, the
\cs{paragraph} would replace the command to create the bullet with
those needed to typeset the paragraph title.

\begin{figure}[p]
\centering
\fbox{\makebox[.975\textwidth][c]{\parbox{\dimexpr.975\textwidth-1in\relax}{
\vspace*{\dimexpr .5in-3.25ex}
\PushPreHook{par}{\llap{\textbullet\enskip}\null}
\paragraph{Lorem ipsum.}
\lipsum[1-4]
\PopPreHook{par}

\vskip.5in
}}}
\caption{Example output.}
\label{fig:example}
\end{figure}

Note that this package is not a panacea. We had to add a \cs{null} to
the |par| hook because \cs{paragraph} uses \cs{lastbox} to remove the
indentation box. Without the \cs{null} it ends up removing the box
constructed by \cs{llap} instead.

Using the post |par| hook solves the \cs{lastbox} problem, but then
the bullet is placed to the right of the \cs{paragraph} title.

Perhaps a better way to solve this problem is to remove the
indentation box first, insert the bullet, and then place the box
after. In this way, the bullet is always to the left of the
paragraph indentation.
\begin{lstlisting}
\PushPreHook{par}{{\setbox0=\lastbox
	\llap{\textbullet\enskip}\box0}}
\end{lstlisting}

\section{Potential pitfalls}
As noted in the previous section, it can be tricky to use the |par|
hook correctly. This section contains an (almost certainly) incomplete
list of pitfalls to watch out for when using \textsf{\jobname}.
\begin{enumerate}
\item When using the |par| hooks, be aware that TeX will insert a box
with the width of \cs{parindent} before the tokens in the pre hook.
One way to handle this is to propogate the box to the right.
\item It is probably not a good idea to use the |hbox|, |vbox|, and
|par| hooks at any place where TeX's output routine is likely to run.
The |excludeor| option \emph{should} help with this, but it might
cause problems with other packages that also modify the output
routine.
\item \LaTeX's kernel takes control of the \cs{everymath} and
\cs{everydisplay} token parameters to make its own adjustments in much
the same way this package does. The \textsf{trace} package uses the
kernel's private macros to insert its own hooks. It is probably best
to only use the post |math| and |display| hooks to ensure that the
kernel has done what it needs to do before you start typesetting stuff
in math mode.
\item When using the |hbox| and |vbox| hooks, any hbox or vbox that
appears in a \cs{setbox} will have the \cs{afterassignment} token
inserted \emph{before} the hooks. This is no different from TeX's
normal behavior with \cs{afterassignment} and
\cs{everyhbox}/\cs{everyvbox}, but can be surprising.
\item I'm sure there are others.
\end{enumerate}

\StopEventually{
	\typeout{**************************************************}
	\typeout{*}
	\typeout{* To finish the installation, you have to move the}
	\typeout{* following file into a directory searched by TeX:}
	\typeout{*}
	\typeout{* \space\space \jobname.sty}
	\typeout{*}
	\typeout{* Documentation is in \jobname.\ifpdf pdf\else dvi\fi.}
	\typeout{*}
	\typeout{* Happy TeXing!}
	\typeout{**************************************************}
	\end{document}
}

\clearpage
\DocInput{\jobname.dtx}
\clearpage
\phantomsection
\addcontentsline{toc}{section}{Change History}%
\PrintChanges
\phantomsection
\addcontentsline{toc}{section}{Index}%
\PrintIndex
\Finale
%</batchfile>
% \fi
% \DoNotIndex{\csdef,\cslet,\csname,\csuse,\def,\endcsname,\endinput,
% \every,\expandafter,\gdef,\global,\ifcsdef,\ifcsempty,\ifdefequal,
% \letcs,\newrobustcmd,\newtoks,\PackageError,\protect,\RequirePackage,
% \space,\the,\undef}
%
% \section{Implementation}
% The package begins with the usual package identification and then it
% declares the one option |excludeor| and finally it loads the
% \textsf{etoolbox} package. This package is not strictly necessary,
% but it does simplify some stuff and provides handy macros for
% dealing with control sequence names.
%
% \changes{v1.0}{2010/09/26}{Initial version}
% \changes{v1.1}{2011/02/03}{Add excludeor}
% \changes{v1.2}{2014/11/26}{Let everyX be redefined}
%    \begin{macrocode}
\NeedsTeXFormat{LaTeX2e}[1999/12/01]
\RequirePackage{svn-prov}
\ProvidesPackageSVN
	{$Id: everyhook.dtx 12 2014-11-26 15:34:56Z steve $}
	[v1.2 \revinfo\ Hooks for low level TeX everyX primitives.]
\DeclareOption{excludeor}{%
	\AtBeginDocument{%
		\output\expandafter{%
			\expandafter\eh@saveclearallhooks
			\the\output
			\eh@restoreallhooks
		}%
	}%
}
\ProcessOptions\relax
\RequirePackage{etoolbox}
%    \end{macrocode}
%
% \begin{macro}{\eh@definehook}
% This performs all of the setup work for each hook. First, it takes
% control of \TeX's token parameter given in the second argument. Then
% it shadows the name of the primitive with a normal token register (and
% copies the current definition). The pre and post hooks are defined
% to be initially empty.
%    \begin{macrocode}
\def\eh@definehook#1#2{%
	\cslet{eh@every#1}#2%
	\newtoks#2%
	\cslet{eh@private#1}#2%
	#2\csuse{eh@every#1}%
	\csdef{eh@pre#1}{}%
	\csdef{eh@post#1}{}%
%    \end{macrocode}
% This is slightly tricky to get right. Basically, we want to set the
% \cs{everyfoo} primitive which we have saved as \cs{eh@everyfoo}
% like
%
% \indent\lstinline!\eh@everyfoo={\eh@prefoo\the\expandafter\everyfoo\eh@postfoo}!.
%
% The reason for the \cs{expandafter} is to make sure it is expanded
% before the the token register \cs{everyfoo} is expanded. Thus if the
% post hook is empty, then code in \cs{everyfoo} sees no additional
% tokens, in case that is important.
%
% Unfortunately, some code wants to redefine \cs{everyfoo} itself in
% order to prevent \emph{other} code that uses \cs{everyfoo} from
% actually setting anything. To deal with that, we use the private
% token list
%
% \indent\lstinline!\eh@everyfoo{\eh@prefoo\the\expandafter\eh@privatefoo\eh@postfoo}!
%    \begin{macrocode}
	\csuse{eh@every#1}\expandafter{\csname eh@pre#1\expandafter\endcsname
		\expandafter\the\csname eh@private#1\expandafter\endcsname
		\csname eh@post#1\endcsname}%
}
%    \end{macrocode}
% \end{macro}
% \begin{macro}{\everypar}
% Define the hooks for the |par| hook.
%    \begin{macrocode}
\eh@definehook{par}\everypar
%    \end{macrocode}
% \end{macro}
% \begin{macro}{\frozen@everymath}
% \begin{macro}{\frozen@everydisplay}
% Define the |math| and |display| hooks. Since the \LaTeX\ kernel has
% already saved \cs{everymath} and \cs{everydisplay} into the frozen
% macros, we take control by redefining the frozen ones instead.
%    \begin{macrocode}
\eh@definehook{math}\frozen@everymath
\eh@definehook{display}\frozen@everydisplay
%    \end{macrocode}
% \end{macro}
% \end{macro}
% \begin{macro}{\everyhbox}
% \begin{macro}{\everyvbox}
% \begin{macro}{\everycr}
% Define the |hbox|, |vbox|, and |cr| hooks and free up some used
% memory.
%    \begin{macrocode}
\eh@definehook{hbox}\everyhbox
\eh@definehook{vbox}\everyvbox
\eh@definehook{cr}\everycr
\undef\eh@definehook
%    \end{macrocode}
% \end{macro}
% \end{macro}
% \end{macro}
%
% \begin{macro}{\eh@hookseparator}
% An separator used to separate tokens in each hook.
%    \begin{macrocode}
\def\eh@hookseparator{}
%    \end{macrocode}
% \end{macro}
% \begin{macro}{\eh@checkhook}
% Check that the hook is one of the six.
%    \begin{macrocode}
\def\eh@checkhook#1#2{%
	\ifcsdef{eh@every#1}{}{\PackageError{everyhook}{Argument #1 to
	\protect#2\space is invalid}{There is no hook for
	\protect\every#1.}}%
}
%    \end{macrocode}
% \end{macro}
% \begin{macro}{\eh@checkhooknotempty}
% Check that the hook is both defined and not empty so that we can
% pop.
%    \begin{macrocode}
\def\eh@checkhooknotempty#1#2#3{%
	\eh@checkhook{#2}#3%
	\ifcsempty{eh@#1#2}{\PackageError{everyhook}{The #1 hook for
	\protect\every#2\space is empty}{I have seen too many
	\protect#3{#2}s.}{}}%
}
%    \end{macrocode}
% \end{macro}
% \begin{macro}{\PushPreHook}
% Prepend tokens to the pre hook, separated via the separator.
%    \begin{macrocode}
\newrobustcmd\PushPreHook[2]{%
	\eh@checkhook{#1}\PushPreHook
	\def\eh@tempi{#2}%
	\letcs\eh@tempii{eh@pre#1}%
	\expandafter\gdef\csname eh@pre#1\expandafter\expandafter
		\expandafter\endcsname\expandafter\expandafter
		\expandafter{\expandafter\eh@tempi\expandafter
		             \eh@hookseparator\eh@tempii}%
	\undef\eh@tempi
	\undef\eh@tempii
}
%    \end{macrocode}
% \end{macro}
% \begin{macro}{\PopPreHook}
% \begin{macro}{\eh@popprehook}
% Check that the hook is not empty, and then pop off the left tokens
% and separator. We can use delimited parameters to strip off the
% first set of tokens.
%    \begin{macrocode}
\newrobustcmd\PopPreHook[1]{%
	\eh@checkhooknotempty{pre}{#1}\PopPreHook
	\expandafter\eh@popprehook\csname eh@pre#1\expandafter
		\expandafter\expandafter\endcsname
		\csname eh@pre#1\endcsname\eh@hookend
}
\def\eh@popprehook#1#2\eh@hookseparator#3\eh@hookend{\gdef#1{#3}}
%    \end{macrocode}
% \end{macro}
% \end{macro}
% \begin{macro}{\PushPostHook}
% Append a separator and tokens to the post hook.
%    \begin{macrocode}
\newrobustcmd\PushPostHook[2]{%
	\eh@checkhook{#1}\PushPostHook
	\letcs\eh@tempi{eh@post#1}%
	\expandafter\gdef\csname eh@post#1\expandafter\endcsname
		\expandafter{\eh@tempi\eh@hookseparator#2}%
	\undef\eh@tempi
}
%    \end{macrocode}
% \end{macro}
% \begin{macro}{\PopPostHook}
% \begin{macro}{\eh@popposthook}
% \begin{macro}{\eh@sentinel}
% Check that the post hook is not empty. Then, iterate over the tokens
% in the list until we reach the end and strip that off.
%    \begin{macrocode}
\newrobustcmd\PopPostHook[1]{%
	\eh@checkhooknotempty{post}{#1}\PopPostHook
	\letcs\eh@tempi{eh@post#1}%
	\expandafter\eh@popposthook\csname eh@post#1\expandafter
		\endcsname\expandafter{\expandafter}\eh@tempi
		\eh@hookend\eh@hookseparator\eh@sentinel\eh@hookend
	\undef\eh@tempi
}
\def\eh@popposthook#1#2\eh@hookseparator#3\eh@hookseparator#4\eh@hookend{%
	\def\eh@tempi{#4}%
	\ifdefequal\eh@sentinel\eh@tempi%
		{\gdef#1{#2}\undef\eh@tempi}%
		{\eh@popposthook#1{#2\eh@hookseparator#3}\eh@hookseparator#4\eh@hookend}%
}
\def\eh@sentinel{\eh@sentinel}
%    \end{macrocode}
% \end{macro}
% \end{macro}
% \end{macro}
% \begin{macro}{\eh@clearhook}
% Internal hook reset.
%    \begin{macrocode}
\def\eh@clearhook#1{%
	\global\csdef{eh@#1}{}%
}
%    \end{macrocode}
% \end{macro}
% \begin{macro}{\ClearPreHook}
% \begin{macro}{\ClearPostHook}
% Reset the pre/post hook to empty.
%    \begin{macrocode}
\newrobustcmd\ClearPreHook[1]{%
	\eh@checkhook{#1}\ClearPreHook
	\eh@clearhook{pre#1}%
}
\newrobustcmd\ClearPostHook[1]{%
	\eh@checkhook{#1}\ClearPostHook
	\eh@clearhook{post#1}%
}
%    \end{macrocode}
% \end{macro}
% \end{macro}
% \begin{macro}{\eh@savehook}
% \begin{macro}{\eh@restorehook}
% Internal macros to \cs{let} the hook to the supplied control
% sequence to save. Perform the \cs{let} in the other direction to
% restore.
%    \begin{macrocode}
\def\eh@savehook#1#2{%
	\letcs#2{eh@#1}%
}
\def\eh@restorehook#1#2{%
	\global\cslet{eh@#1}#2%
}
%    \end{macrocode}
% \end{macro}
% \end{macro}
% \begin{macro}{\SavePreHook}
% \begin{macro}{\SavePostHook}
% \begin{macro}{\RestorePreHook}
% \begin{macro}{\RestorePostHook}
% User macros to save and restore hooks.
%    \begin{macrocode}
\newrobustcmd\SavePreHook[2]{%
	\eh@checkhook{#1}\SavePreHook
	\eh@savehook{pre#1}#2%
}
\newrobustcmd\SavePostHook[2]{%
	\eh@checkhook{#1}\SavePostHook
	\eh@savehook{post#1}#2%
}
\newrobustcmd\RestorePreHook[2]{%
	\eh@checkhook{#1}\RestorePreHook
	\eh@restorehook{pre#1}#2%
}
\newrobustcmd\RestorePostHook[2]{%
	\eh@checkhook{#1}\RestorePostHook
	\eh@restorehook{post#1}#2%
}
%    \end{macrocode}
% \end{macro}
% \end{macro}
% \end{macro}
% \end{macro}
% \begin{macro}{\eh@saveclearallhooks}
% \begin{macro}{\eh@restoreallhooks}
% Internal macros to save and clear (resp.\ restore) all hooks at the
% start (resp.\ end) of the output routine.
%    \begin{macrocode}
\def\eh@saveclearallhooks{%
	\global\eh@savehook{prepar}\eh@or@prepar
	\global\eh@savehook{postpar}\eh@or@postpar
	\global\eh@savehook{premath}\eh@or@premath
	\global\eh@savehook{postmath}\eh@or@postmath
	\global\eh@savehook{predisplay}\eh@or@predisplay
	\global\eh@savehook{postdisplay}\eh@or@postdisplay
	\global\eh@savehook{prehbox}\eh@or@prehbox
	\global\eh@savehook{posthbox}\eh@or@posthbox
	\global\eh@savehook{prevbox}\eh@or@prevbox
	\global\eh@savehook{postvbox}\eh@or@postvbox
	\global\eh@savehook{precr}\eh@or@precr
	\global\eh@savehook{postcr}\eh@or@postcr
	\eh@clearhook{prepar}%
	\eh@clearhook{postpar}%
	\eh@clearhook{premath}%
	\eh@clearhook{postmath}%
	\eh@clearhook{predisplay}%
	\eh@clearhook{postdisplay}%
	\eh@clearhook{prehbox}%
	\eh@clearhook{posthbox}%
	\eh@clearhook{prevbox}%
	\eh@clearhook{postvbox}%
	\eh@clearhook{precr}%
	\eh@clearhook{postcr}%
}
\def\eh@restoreallhooks{%
	\eh@restorehook{prepar}\eh@or@prepar
	\eh@restorehook{postpar}\eh@or@postpar
	\eh@restorehook{premath}\eh@or@premath
	\eh@restorehook{postmath}\eh@or@postmath
	\eh@restorehook{predisplay}\eh@or@predisplay
	\eh@restorehook{postdisplay}\eh@or@postdisplay
	\eh@restorehook{prehbox}\eh@or@prehbox
	\eh@restorehook{posthbox}\eh@or@posthbox
	\eh@restorehook{prevbox}\eh@or@prevbox
	\eh@restorehook{postvbox}\eh@or@postvbox
	\eh@restorehook{precr}\eh@or@precr
	\eh@restorehook{postcr}\eh@or@postcr
}
\endinput
%    \end{macrocode}
% \end{macro}
% \end{macro}
\endinput
